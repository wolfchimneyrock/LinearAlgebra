\documentclass{letter}
\usepackage{enumitem}
\usepackage{mathtools}
\usepackage{fancyhdr}
\usepackage{xcolor}
\usepackage{mdframed}
\usepackage{bm}
\usepackage[letterpaper,portrait,left=2cm,right=2cm,top=3.5cm,bottom=2cm]{geometry}
\pagestyle{fancy}

\fancyhf{}
\rhead{Robert Wagner\\June 7, 2016}
\lhead{Math 2101\\Assignment 2}
\newcounter{question}
\setcounter{question}{0}
\usepackage{amsmath,amsthm}
\usepackage{amssymb}

% magnitude bars
\newcommand{\norm}[1]{\lvert #1 \rvert}

% explicit vector
\newcommand{\Ve}[1]{\langle #1 \rangle}

% named vector
\newcommand{\Vn}[1]{\vec{\bm{#1}}}

% a line with an arrow above
\newcommand{\Line}[1]{\overrightarrow{#1}}

\newcommand\Que[1]{%
   \leavevmode\noindent
   #1
}

\newcommand\Ans[2][]{%
   \leavevmode\noindent
   {
       \begin{mdframed}[backgroundcolor=blue!10]
       #2
       \end{mdframed}
   }
}

% this is like align but squashed     
\newenvironment{salign}
 {\par$\!\aligned}
 {\endaligned$\par}

% a matrix, parameter is column count
\newenvironment{Mat}[1]{%
  \left[\begin{array}{*{#1}{c}}
}{%
  \end{array}\right]
}

% an augmented matrix, parameter is non-augmented column count
\newenvironment{Amat}[1]{%
  \left[\begin{array}{@{}*{#1}{r}|r@{}}
}{%
  \end{array}\right]
}

\begin{document}


\begin{enumerate}
    \item Produce the row echelon form of the augmented matrix for the given system of equations.  Specify the elementary row operations used to produce the matrix.  Then give a (possibly parameterized) expression for all solutions in vector form, and identify whether the solution(s) corresponds to a point, line, or plane.  If a solution does not exist, say so.  Finally, give the rank of each coefficient matrix.
    \begin{enumerate}[label=(\alph*)]
        \item\Que{The system of equations:
        \begin{minipage}[t]{0.25\textwidth}
        \begin{flushright}
        $8x+5y=25$\par $3x-7y=10$
        \end{flushright}
        \end{minipage}
        }
        \Ans{\begin{salign}
         & ~~ 3R_1-8R_2 \to R_2 ~~~~ 213R_1-15R_2 \to R_1 ~~~~~\text{normalize}~~ \\
            \begin{Amat}{2}
              8 &  5 & 25 \\
              3 & -7 & 10
            \end{Amat}
            &=
            \begin{Amat}{2}
              8 &  5 & 25 \\
              0 & 71 & -5
            \end{Amat}
            =
            \begin{Amat}{2}
              1704 &  0 & 5400 \\
              0    & 71 & -5
            \end{Amat}
            =
            \begin{Amat}{2}
              1 & 0 & \frac{225}{71} \\
              0 & 1 & \frac{-5}{71}
            \end{Amat}
           \\
           &= \text{ the point }\left(\frac{225}{71},\frac{-5}{71}\right) \in \mathbb{R}^2 \text{ with rank = 2}
           \end{salign}
        }    
        \item \Que{The system of equations:
        \begin{minipage}[t]{0.25\textwidth}
        \begin{flushright}
        $2x+3y-4z=11$\\
        $x+5y-2z=10$\\
        $4x-3y-z=25$
        \end{flushright}
        \end{minipage}
        }
        \Ans{
            \begin{salign}
              & ~~~~~~ R_2-2R_1 \to R_2 ~~~~~~~~~ 9R_1+5R_3 \to R_1~~~~~~~~~~~~ \\
              & ~~~~~~ R_3-2R_2 \to R_3 ~~~~~~~~~ 9R_2-7R_3 \to R_3~~~~~~~~~~~ 49R_1+17R_3 \to R_1\\
              \begin{Amat}{3}
                1 &  5 & -2 & 10 \\
                2 &  3 & -4 & 11 \\
                4 & -3 & -1 & 25
              \end{Amat}
              &=
              \begin{Amat}{3}
                1 &  5 & -2 & 10 \\
                0 & -7 &  0 &  9 \\
                0 & -9 &  7 &  3    
              \end{Amat}
              =
              \begin{Amat}{3}
                9 &  0 &  17 & 105 \\
                0 & -7 &   0 &   9 \\
                0 &  0 & -49 &  60
              \end{Amat}
              =
              \begin{Amat}{3}
                441 &  0 &   0 & 6165 \\
                  0 & -7 &   0 &    9 \\
                  0 &  0 & -49 &   60
              \end{Amat} 
              \\ 
              & ~~~~~~ \text{normalize} \\
              &=
              \begin{Amat}{3}
                1 & 0 & 0 & \frac{685}{49} \\
                0 & 1 & 0 & \frac{-9}{7}   \\
                0 & 0 & 1 & \frac{-60}{49}
              \end{Amat}
              = \text{ the point }\left(\frac{685}{49},\frac{-9}{7},\frac{-60}{49}\right) \in \mathbb{R}^3 \text{ with rank = 3}
            \end{salign}
        }
        \item \Que{The system of equations:
        \begin{minipage}[t]{0.25\textwidth}
        \begin{flushright}
        $7x+5y=11$\\
        $2x-4y=3$\\
        $3x-2y=15$
        \end{flushright}
        \end{minipage}
        }
        \Ans{
          \begin{salign}
            & ~~~~ 2R_1-7R_2 \to R_2 \\
            & ~~~~ 3R_1-7R_3 \to R_3~~~~~ \text{normalize}\\
            \begin{Amat}{2}
              7 &  5 & 11 \\
              2 & -4 & 3  \\
              3 & -2 & 15
            \end{Amat}
            &=
            \begin{Amat}{2}
              7 &  5 &  11 \\
              0 & 38 &   1 \\
              0 & 29 & -72 
            \end{Amat}
            =
            \begin{Amat}{2}
              7 &  5 &  11 \\
              0 &  1 & \frac{1}{38} \\
              0 &  1 & \frac{-72}{29}
            \end{Amat}
            \\
            & \text{and we have reached a contradiction since }
            y=\frac{1}{38} 
            \text{ and }
            y=\frac{-72}{29}
            \text{ cannot both be true.}\\
            & \text{Therefore this system has no solution.}
          \end{salign}
        }
        \item \Que{The system of equations:
        \begin{minipage}[t]{0.25\textwidth}
        \begin{flushright}
        $x+3y-2z=25$\\
        $2x-5y+4z=10$
        \end{flushright}
        \end{minipage}
        }
        \Ans{
          \begin{salign}
            & ~~~~~~~~ R_2-2R_1 \to R_2 ~~~~~~~~~ 11R_1+3R_2 \to R_1 ~~~~~~~~~~ \text{normalize}\\
            \begin{Amat}{3}
              1 &  3 & -2 & 25 \\
              2 & -5 &  4 & 10
            \end{Amat}
            &=
            \begin{Amat}{3}
              1 &   3 & -2 &  25 \\
              0 & -11 &  8 & -40
            \end{Amat}
            =
            \begin{Amat}{3}
              11 &   0 & 2 & 155 \\
               0 & -11 & 8 & -40
            \end{Amat}
            =
            \begin{Amat}{3}
               1 & 0 & \frac{2}{11}  & \frac{155}{11} \\
               0 & 1 & \frac{-8}{11} & \frac{40}{11} 
            \end{Amat}
            \\
            &= \text{ the line } L(s)=\left(\frac{155}{11},\frac{40}{11},0\right) + s\Ve{-2,8,11} 
               \text{ for all } s \in \mathbb{R} \\
            &  \text{matrix rank } = 2
          \end{salign}
        }
        \newpage
        \item \Que{The system of equations:
        \begin{minipage}[t]{0.25\textwidth}
        \begin{flushright}
        $2x-4y+z=10$\\
        $4x-8y+2z=25$
        \end{flushright}
        \end{minipage}
        }
        \Ans{
          \begin{salign}
            & ~~~~ R_2-2R_1 \to R_2\\
            \begin{Amat}{3}
              2 & -4 & 1 & 10 \\
              4 & -8 & 2 & 25
            \end{Amat}
            &=
            \begin{Amat}{3}
              2 & -4 & 1 & 10 \\
              0 &  0 & 0 &  5
            \end{Amat}
            & \text{and we have reached a system with no solutions.}
          \end{salign}
        }
        \item \Que{The system of equations:
        \begin{minipage}[t]{0.25\textwidth}
        \begin{flushright}
        $2x-4y+z=5$\\
        $4x-8y+2z=30$
        \end{flushright}
        \end{minipage}
        }
        \Ans{
          \begin{salign}
            & ~~~~ R_2-2R_1 \to R_2 \\
            \begin{Amat}{3}
              2 & -4 & 1 & 10 \\
              4 & -8 & 2 & 30
            \end{Amat}
            &=
            \begin{Amat}{3}
              2 & -4 & 1 & 10 \\
              0 &  0 & 0 & 10
            \end{Amat}
            & \text{and we have reached a system with no solutions.}
          \end{salign}
        }
    \end{enumerate}
    ~\\
    \item 
    Let $\Vn{u} = \Ve{ u_1, u_2, u_3 }$ and $\Vn{v}=\Ve{ v_1, v_2, v_3 }$.  
    We want to find a vector perpendicular to both $\Vn{u}$ and $\Vn{v}$.
    \begin{enumerate}[label=(\alph*)]
    \item \Que{
        Let the components of this vector be $\Ve{x_1,x_2,x_3}$.  
        Write down a system of equations corresponding to the requirement that this vector 
        be perpendicular to $\Vn{u}$ and $\Vn{v}$.
    }
    \Ans{
      For any two vectors $\Vn{a}$ and $\Vn{b}$ to be perpendicular, $\Vn{a}\cdot\Vn{b}=0$ must be true.
      \begin{align*}
        \Vn{u} \cdot \Vn{x} &= 0 \\
        \Vn{v} \cdot \Vn{x} &= 0
        \shortintertext{which can be stated as}
        u_1x_1+u_2x_2+u_3x_3 &=0 \\
        v_1x_1+v_2x_2+v_3x_3 &=0
      \end{align*}
    }
    \item \Que{
        What is the rank of the corresponding coefficient matrix; what does this say about the number of vectors that are perpendicular to both $\Vn{u}$ and $\Vn{v}$; and why is this surprising?
    }
    \Ans{
    \begin{salign}
        & \frac{R_2}{v_1}-\frac{R_1}{u_1} \to R_2 ~~~ \frac{R_1}{u_1}-\frac{R_2u_1u_2v_1}{u_1(u_1v_2-u_2v_1)} \to R_1 ~~~ \frac{R_2u_1v_1}{u_1v_2-u_2v_1} \to R_2 \\
        \begin{Amat}{3}
            u_1 & u_2 & u_3 & 0 \\
            v_1 & v_2 & v_3 & 0
        \end{Amat}
        &=
        \begin{Amat}{3}
            u_1 &                   u_2        &                    u_3       & 0 \\
            0   & \frac{u_1v_2-u_2v_1}{u_1v_1} & \frac{u_1v_3-u_3v_1}{u_1v_1} & 0
        \end{Amat}
        =
        \begin{Amat}{3}
            1 & 0 & \frac{u_3v_2-u_2v_3}{u_1v_2-u_2v_1} & 0 \\
            0 & 1 & \frac{u_1v_3-u_3v_1}{u_1v_2-u_2v_1} & 0  
        \end{Amat}
        \\
        &= (0,0,0) + s\Ve{u_2v_3-u_3v_2, u_3v_1-u_1v_3, u_1v_2-u_2v_1} \text{ for all } s \in \mathbb{R}
    \end{salign}
    matrix rank = 2\\
    The solution is a line, thus there are an infinite number of vectors perpendicular to both $\Vn{u}$ and $\Vn{v}$.
    }
    \item \Que{
        Find a vector perpendicular to $\Vn{u}$ and to $\Vn{v}$, whose components are \textit{not} fractional expressions of the components of $\Vn{u}$ and $\Vn{v}$.  (It's possible the components are fractions or even real numbers; what you want to avoid is having a component like $\frac{u_1}{v_3}$)
    }
    \Ans{
    Trivial answer: $\Vn{z} = \Ve{0,0,0}$ is perpendicular to both $\Vn{u}$ and $\Vn{v}$
    since $\Vn{u}\cdot\Vn{z}=0$ and $\Vn{v}\cdot\Vn{z}=0$ are true. \\
    General answer: $\Vn{x} = \Ve{u_2v_3-u_3v_2, u_3v_1-u_1v_3, u_1v_2-u_2v_1}$ is perpendicular to both
    $\Vn{u}$ and $\Vn{v}$ as shown in the answer above.
    }
    \item \Que{
        Find a vector perpendicular to $\Ve{1,1,-1}$ and $\Ve{2,1,-3}$.  (This is called a \textbf{normal vector})
    }
    \Ans{
        $\Vn{n}=\Ve{(1)(-3)-(-1)(1),(-1)(2)-(1)(-3),(1)(1)-(1)(2)} = \underline{\Ve{-2,1,-1}}$\\
        verify: \\
        $\Ve{1,1,-1}\cdot\Ve{-2,1,-1} = (1)(-2)+(1)(1)+(-1)(-1) = -2+1+1 = 0 \\
         \Ve{2,1,-3}\cdot\Ve{-2,1,-1} = (2)(-2)+(1)(1)+(-3)(-1) = -4+1+3 = 0
        $ 
    }
    
    \item \Que{
        Let $X=(x,y,z)$ be a point in the plane containing the points $P=(1,2,4)$, $Q=(-3,-1,2)$, and $R=(1,1,0)$.
        Write the equation of the plane.  Suggestion:  The vector perpendicular to $\Line{PQ}$ and $\Line{PR}$ will also be perpendicular to $\Line{PX}$.
    }    
    \Ans{
    Let 
    \begin{align*}
        \Vn{u}&=\Line{PQ}=\Ve{-4,-3,-2},\\
        \Vn{v}&=\Line{PR}=\Ve{ 0,-1,-4}, \\
        \Vn{x}&=\Line{PX}=\Ve{x-1,y-2,z-4}, \\
        \Vn{n} &=\Vn{u}\times\Vn{v}\\
                &=\Ve{(-3)(-4)-(-2)(-1),(-2)(0)-(-4)(-4),(-4)(-1)-(-3)(0)}\\
                &=\Ve{10,-16,4}
        \end{align*}
    Since we know that $\Vn{n}$ is perpendicular to both $\Vn{u}$ and $\Vn{v}$, 
    and that subsequently $\Vn{n}$ will be perpendicular to $\Line{PX}$ for any point $X$ on the plane,  
    to specify the plane we can solve the equation:
    \begin{align*}
        0 &= \Vn{n}\cdot\Vn{x} \\
          &= 10(x-1)-16(y-2)+4(z-4) \\
          &= 10x-16y+4z+6 \\
          &= \underline{5x-8y+2z=-3}
    \end{align*}
    And we have found an equation for the plane.
    \begin{align*}
    \text{Verify:  } &P: 5(1)-8(2)+2(4)  &&= 5-16+8&=-3 ~~OK \\
    \text{Verify:  } &Q: 5(-3)-8(-1)+2(2)&&=-15+8+4&=-3 ~~OK \\
    \text{Verify:  } &R: 5(1)-8(1)+2(0)  &&=5-8    &=-3 ~~OK
    \end{align*}
    }
    \item \Que{
        Find the distance between the point $(2,1,0)$ and the plane $PQR$.
    }
    \Ans{
    Let 
    \begin{align*}
             A&=(2,1,0) \text{ be the point of interest,} \\
        \Vn{n}&=\Ve{5,-8,2} \text{ be a normal vector of the plane,} \\
        \Vn{p}&=\Line{PA}=\Ve{1,-1,-4}
    \shortintertext{To find the distance, we solve the equation:}
        D &= \frac{\norm{\Vn{n}\cdot\Vn{p}}}{\norm{\Vn{n}}}
          = \frac{\norm{5+8-8}}{\sqrt{25+64+4}} 
          = \frac{5}{\sqrt{93}}\approx\underline{0.5185}
    \end{align*}
    }
    \end{enumerate}
    \newpage
    \item An important use of matrices is \textbf{stochastic modeling}.  An example is the following:  Imagine a park with three locations: a lake, a picnic area, and a playground.  Every hour, on the hour, the parkgoers move according to the following rules:
    \begin{itemize}
      \item Half of those at the lake move to the picnic area, and one-quarter of those at the lake move to the playground.
      \item Half of those at the picnic area move to the lake, and the other half go to the playground.
      \item Half of those at the playground go to the picnic area.
    \end{itemize}
    Once they arrive, the parkgoers stay until the next hour, at which point they move again according to the same rules.
    \begin{enumerate}[label=(\alph*)]
    \item \Que{Suppose there are 100 persons at each location.  How many persons are there at each of the locations one hour later?  Suggestion: in life it is often easier to determine where you've come \textit{from} than it is to determine where you're going \textit{to}.}
    \Ans{
      \makebox[2cm][l]{picnic:} \makebox[3cm][l]{0 (stayed)} + 
        \makebox[3cm][l]{50 (from lake)} + \makebox[3cm][l]{50 (from playgr.)} = 100\\
      \makebox[2cm][l]{lake:} \makebox[3cm][l]{50 (from picnic)} + 
         \makebox[3cm][l]{25 (stayed)} + \makebox[3cm][l]{0 (from playgr.)} = 75\\ 
      \makebox[2cm][l]{playground:} \makebox[3cm][l]{50 (from picnic)} + 
         \makebox[3cm][l]{25 (from lake)} + \makebox[3cm][l]{50 (stayed)} = 125
    }
    \item \Que{Let $p_n$, $l_n$, $g_n$ be variables representing the number of persons at the picnic area, lake, and playground at hour $n$.  write down equations for $p_{n+1}$, $l_{n+1}$, $g_{n+1}$, giving the number of persons at the picnic area, lake, and playground at hour $n+1$.}
    \Ans{
        \begin{salign}
            p_{n+1} &= 0p_n + (1/2)l_n + (1/2)g_n \\
            l_{n+1} &= (1/2)p_n + (1/4)l_n + 0g_n \\
            g_{n+1} &= (1/2)p_n + (1/4)l_n + (1/2)g_n 
        \end{salign}
    }
    \item \Que{Write down the coefficient matrix $T$ for the system of equations you just wrote.  }
    \Ans{
       $ T = 
       \begin{Mat}{3}
         0   & 1/2 & 1/2  \\
         1/2 & 1/4 & 0    \\
         1/2 & 1/4 & 1/2
       \end{Mat}
       $
    }
    \item \Que{
    One of the following could be a stochastic matrix and the other cannot.  Identify the matrix that cannot be a stochastic matrix and explain why.  Also give the corresponding movement rules for the valid matrix.
    
    \[
    A = \begin{Mat}{3}
            1/2 & 1/4 & 1/2 \\
            1/2 & 1/4 & 1/2 \\
            0   & 1/2 & 1/4
          \end{Mat} 
    \text{  } 
    B = \begin{Mat}{3}
            1/4 &   0 & 1/3 \\
            1/2 & 1/2 & 1/6 \\
            1/4 & 1/2 & 1/2
        \end{Mat}
    \]
    } 
    \Ans{matrix $A$ cannot be a stochastic matrix because the $g_n$ column doesn't sum to one; it would basically be saying that people from the playground go to two places simultaneously.  \\
    The rules for matrix B:
    \begin{itemize}
      \item 1/4 of those at the picnic area go to the playground, and 1/2 of those at the picnic go to the lake
      \item 1/2 of those at the lake go to the playground
      \item 1/3 of those at the playground go to the picnic, and 1/6 at the playground go to the lake
    \end{itemize}
    }
    \item \Que{Generalize your observation: A matrix $M$ can be a stochastic model provided ...}
    \Ans{
    Each column sums to one since every member of the population present at $t=n$ must be present at $t=n+1$.
    Each row doesn't necessarily have to sum to one since the population can move between unique places. 
    The matrix should be square since the row and columns both correspond to the number of unique places.
    }
    \end{enumerate}
    \newpage
    \item Another use of matrices is for \textbf{discrete time modeling}.  Consider the following problem:\\
    Suppose you have a pair of rabbits that breed according to the following rules:
      \begin{itemize}
        \item Rabbits mature after two months, and will produce a pair of rabbits every month after.
        \item The pair of rabbits are always male/female, and mature rabbits will always find a mate.
        \item No rabbits die.
      \end{itemize}
    Let $x_n$ be the number of pairs of immature rabbits at the end of month $n$, and let $y_n$ be the number of pairs of mature rabbits at the end of month $n$.
    \begin{enumerate}[label=(\alph*)]
    \item \Que{What are $x_0$ and $y_0$?}
    \Ans{$x_0=1$\ and $y_0=0$}
    \item \Que{Find $x_1$ and $y_1$.}
    \Ans{$x_1=0$\ and $y_1=1$}
    \item \Que{Suppose you know $x_n$ and $y_n$.  How would you find $x_{n+1}$ and $y_{n+1}$?}
    \Ans{$x_{n+1}=y_n$\ and $y_{n+1}=x_n+y_n$}
    \item \Que{Treating $x_n$ and $y_n$ as our twin variables, write down the coefficient matrix for the system of equations that give the values for $x_{n+1}$ and $y_{n+1}$.  }
    \Ans{
      $
      \begin{Mat}{2}
        x_{n+1} &= 0x_n + 1y_n \\
        y_{n+1} &= 1x_n + 1y_n
      \end{Mat}
      =
      \begin{Mat}{2}
        0 & 1 \\
        1 & 1
      \end{Mat}
      $
      }
    \item \Que{Note that this matrix cannot be a stochastic matrix.  What essential difference exists between the rabbit problem and the park problem?}
    \Ans{The park problem is a ``zero sum game" where the population is fixed, therefore the columns must each sum to one. The rabbit problem involves a growth in population, therefore it would be expected that at least one column sums to greater than one. }
    \end{enumerate}
    \newpage
    \item Answer the following questions.  Let $\Vn{v}=\Ve{2,1,3,4}$ and $\Vn{u}=\Ve{1,1,-1,2}$.
    \begin{enumerate}[label=(\alph*)]
    \item \Que{Find the angle $\theta$ between $\Vn{v}$ and $\Vn{u}$.}
    \Ans{
      \begin{align*}
        \cos{\theta} &= \frac{\Vn{u}\cdot\Vn{v}}{\norm{\Vn{u}}\norm{\Vn{v}}}
                      = \frac{2+1-3+8}{\sqrt{30}\sqrt{7}}=\frac{8}{\sqrt{210}} \\        
        \arccos{\frac{8}{\sqrt{210}}} &\approx 0.98597\text{(rad)} \approx \underline{56.49^{\circ}}
      \end{align*}
    }
    \item \Que{Find a unit vector (a vector with length 1) that goes in the same direction as $\Vn{v}$.}
    \Ans{
         $
         \frac{1}{\norm{\Vn{v}}}\Vn{v}=\frac{1}{\sqrt{30}}\Ve{2,1,3,4}
         = \underline{\Ve{\frac{2}{\sqrt{30}},\frac{1}{\sqrt{30}},\frac{3}{\sqrt{30}},\frac{4}{\sqrt{30}}}} 
         $    
    }
    \item \Que{Find $3\Vn{v}-2\Vn{u}$.}
    \Ans{
        $3\Vn{v}-2\Vn{u}=3\Ve{2,1,3,4}-2\Ve{1,1,-1,2}=\Ve{6,3,9,12}-\Ve{2,2,-2,4}=\underline{\Ve{4,1,11,8}}$    
    }
    \item \Que{Find $a,b$ such that $a\Vn{v}+b\Vn{u}$ is perpendicular to $3\Vn{v}-2\Vn{u}$.}
    \Ans{
      \begin{align*}
        0&=(a\Vn{v}+b\Vn{u})\cdot\Ve{4,1,11,8} \\
         &=\Ve{2a+b,a+b,3a-b,4a+2b}\cdot\Ve{4,1,11,8} \\
         &=4(2a+b)+(a+b)+11(3a-b)+8(4a+2b) \\
         &=8a+4b+a+b+33a-11b+32a+16b \\
         &=74a+10b \\
        b&=-\frac{74a}{10} 
      \end{align*}
      Therefore a solution could be $a=10$ and $b=-74$. \\
      Verify: 
      \begin{align*}
      10\Vn{v}-74\Vn{u} &= 10\Ve{2,1,3,4}-74\Ve{1,1,-1,2} \\
                        &= \Ve{20-74,10-74,30+74,40-148}  \\
                        &= \Ve{-54,-64,104,-108} \\
      \Ve{-54,-64,104,-108}\cdot\Ve{4,1,11,8} &= (-54)(4)+(-64)(1)+(104)(11)+(-108)(8) \\
                                              &= (-216) + (-64) + (1144) + (-864) = 0 ~~ \text{OK} 
      \end{align*}
    }
    \end{enumerate}
\end{enumerate} 
\end{document}  