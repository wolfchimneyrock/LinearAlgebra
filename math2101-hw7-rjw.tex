\documentclass{letter}
\usepackage{enumitem}
\usepackage{mathtools}
\usepackage{fancyhdr}
\usepackage{xcolor}
\usepackage{mdframed}
\usepackage{bm}
\usepackage[letterpaper,portrait,left=2cm,right=2cm,top=3.5cm,bottom=2cm]{geometry}
\pagestyle{fancy}

\fancyhf{}
\rhead{Robert Wagner\\July 2, 2016}
\lhead{Math 2101\\Assignment 7}
\newcounter{question}
\setcounter{question}{0}
\usepackage{amsmath,amsthm}
\usepackage{amssymb}
\usepackage{tikz}
\usetikzlibrary{arrows}


% magnitude bars
\newcommand{\norm}[1]{\lvert #1 \rvert}

% explicit vector
\newcommand{\Ve}[1]{\langle #1 \rangle}

% named vector
\newcommand{\Vn}[1]{\vec{#1}}

% a line with an arrow above
\newcommand{\Line}[1]{\overrightarrow{#1}}

% equals with question mark above
\newcommand{\?}{\stackrel{?}{=}}

% formatting for questions
\newcommand\Que[1]{%
   \leavevmode\noindent
   #1
}

% formatting for answers
\newcommand\Ans[2][]{%
   \leavevmode\noindent
   {
       \begin{mdframed}[backgroundcolor=blue!10]
       #2
       \end{mdframed}
   }
}

% this is like align but squashed     
\newenvironment{salign}
 {\par$\!\aligned}
 {\endaligned$\par}

% a matrix, parameter is column count
\newenvironment{Mat}[1]{%
  \left[\begin{array}{*{#1}{r}}
}{%
  \end{array}\right]
}


% a matrix, parameter is column count centered
\newenvironment{Cmat}[1]{%
  \left[\begin{array}{*{#1}{c}}
}{%
  \end{array}\right]
}

% an augmented matrix, with one augmented column
\newenvironment{Amat}[1]{%
  \left[\begin{array}{@{}*{#1}{r}|r@{}}
}{%
  \end{array}\right]
}
% an augmented matrix with n columns and n augmented columns
\newenvironment{Amat2}[1]{%
  \left[\begin{array}{@{~}*{#1}{r}| @{~~}*{#1}{r}}
}{%
  \end{array}\right]
}
% an augmented matrix with n columns and m augmented columns
\newenvironment{Amat3}[2]{%
  \left[\begin{array}{@{~}*{#1}{r}| @{~~}*{#2}{r}}
}{%
  \end{array}\right]
}


\begin{document}


\begin{enumerate}
    \item Find the determinant of each matrix.
    \begin{enumerate}
    \item $\begin{Mat}{2} 2 & 3 \\ 1 & -1 \end{Mat}$
    \Ans{
       $(2\cdot -1)-(3\cdot 1) = -2 - 3 = \underline{-5}$
    }
    \item $\begin{Mat}{2} 2 & -3 \\ 4 & -6 \end{Mat}$ 
    \Ans{
       $(2\cdot -6)-(-3\cdot 4) = -12+12 = \underline{0}$
    }    
    \item $\begin{Mat}{3} 3 & -1 &  1 \\ 
                          1 &  0 & -1 \\ 
                          2 &  1 &  1 \end{Mat}$ 
    \Ans{
       $(1)(-1-1)(-1)^3 + 0 + (-1)(3--2)(-1)^5 = 2 + 5 = \underline{7}$
    }    
    \end{enumerate}
    ~\\     
    \item 
       Every mathematical formula is a summary of all the steps of an algorithm.  Cramer's Rule is a formula that summarizes all steps necessary to solve a system of $n$\ equations with $n$\ unknowns.
    \begin{enumerate}[label=(\alph*)]
    \item \Que{
        Consider a system of two equations with two unknowns:
        \begin{align*}
            a_{11}x+a_{12}y &= c_1 \\
            a_{21}x+a_{22}y &= c_2
        \end{align*}       
        Solve this system to produce formulas for the values $x$\ and $y$.
    }
    \Ans{
       \begin{align*}
           \begin{Amat}{2} a_{11} & a_{12} & c_1 \\
                           a_{21} & a_{22} & c_2 \end{Amat}
           &\to
           \begin{Amat}{2} a_{11} &                                a_{12} & c_1 \\
                                0 & \frac{a_{11}a_{22}-a_{12}a_{21}}{a_{11}} & c_2 - \frac{c_1a_{21}}{a_{11}} \end{Amat} \\
           &\to
           \begin{Amat}{2} a_{11} & 0 & c_1-\left(\frac{a_{11}a_{12}}{a_{11}a_{22}-a_{12}a_{21}}\right)\left(\frac{c_2a_{11}-c_1a_{21}}{a_{11}}\right) \\
                                0 & \frac{a_{11}a_{22}-a_{12}a_{21}}{a_{11}} &\frac{c_2a_{11}-c_1a_{21}}{a_{11}} \end{Amat} \\
           &\to
           \begin{Amat}{2} a_{11} & 0 & \frac{c_1a_{11}a_{22}-c_1a_{12}a_{21}}{a_{11}a_{22}-a_{12}a_{21}} - \frac{c_2a_{11}a_{12}-c_1a_{12}a_{21}}{a_{11}a_{22}-a_{12}a_{21}} \\
                                0 & 1 & \left(\frac{c_2a_{11}-c_1a_{21}}{a_{11}}\right)\left(\frac{a_{11}}{a_{11}a_{22}-a_{12}a_{21}}\right) \end{Amat} \\
           &\to
           \begin{Amat}{2} 1 & 0 & \frac{c_1a_{22}-c_2a_{12}}{a_{11}a_{22}-a_{12}a_{21}} \\
                           0 & 1 & \frac{c_2a_{11}-c_1a_{21}}{a_{11}a_{22}-a_{12}a_{21}} \end{Amat} \\
           x &= \frac{c_1a_{22}-c_2a_{12}}{a_{11}a_{22}-a_{12}a_{21}} \\
           y &= \frac{c_2a_{11}-c_1a_{21}}{a_{11}a_{22}-a_{12}a_{21}}
       \end{align*}
    }
    \newpage
    \item \Que{
        Let $A$\ be the coefficient matrix; $A_x$\ be the matrix produced by replacing first column of the coefficient matrix (corresponding to $x$) with the constant vector; $A_y$\ be the matrix produced by replacing the second column of the coefficient matrix (corresponding to $y$) with the constant vector.  Find $\det A$, $\det A_x$, and $\det A_y$.
    }
    \Ans{
    \begin{align*}
        A &= \begin{Mat}{2}  a_{11} & a_{12} \\ a_{21} & a_{22} \end{Mat} &\det A   = a_{11}a_{22} - a_{12}a_{21} \\
        A_x &= \begin{Mat}{2}   c_1 & a_{12} \\    c_2 & a_{22} \end{Mat} &\det A_x = c_1a_{22} - c_2a_{12} \\
        A_y &= \begin{Mat}{2}a_{11} &    c_1 \\ a_{21} &   c_2  \end{Mat} &\det A_y = c_2a_{11} - c_1a_{21}  
    \end{align*}
        
    }
    \item \Que{
        Express the values of $x$\ and $y$\ in terms of these determinants.
    }
    \Ans{
    \begin{align*}
        &x = \frac{\det A_x}{\det A}
        &y = \frac{\det A_y}{\det A}
    \end{align*}
       
    }
    \item \Que{
      Use Cramer's Rule to solve:
      \begin{align*}
          3x+4y &= 9 \\
          7x-27 &= 8
      \end{align*}
    }
    \Ans{
      \begin{align*}
        x &= \frac{(9\cdot -27)-(4\cdot 8)}{(3\cdot 8)-(4\cdot 7)} = \frac{-243-32}{-4} &= 68\frac{3}{4}\\
        y &= \frac{(3 \cdot 8)-(9 \cdot 7)}{(3\cdot 8)-(4\cdot 7)} = \frac{32-63}{-4} &= 7\frac{3}{4}
      \end{align*}
    }
    \item \Que{
       Cramer's Rule generalizes (with $A_z$\ defined as you'd expect it to be).  Use it to solve:
       \begin{align*}
            x+3y-4z &= 8 \\
           2x-4y-7z &= 1 \\
            x-2y+8  &= 0
       \end{align*}
    }
    \Ans{
    \begin{align*}
        A  &= \begin{Mat}{3} 1 &  3 & -4 \\ 
                             2 & -4 & -7 \\  
                             1 & -2 &  0 \end{Mat} ~~
        A_x = \begin{Mat}{3} 8 &  3 & -4 \\ 
                             1 & -4 & -7 \\ 
                            -8 & -2 &  0 \end{Mat} ~~
        A_y = \begin{Mat}{3} 1 &  8 & -4 \\ 
                             2 &  1 & -7 \\  
                             1 & -8 &  0 \end{Mat} ~~
        A_z = \begin{Mat}{3} 1 &  3 &  8 \\ 
                             2 & -4 &  1 \\  
                             1 & -2 & -8 \end{Mat} \\
        \det A   &= (-4)(-4--4)(-1)^{4} + (-7)(-2-3)(-1)^{5} + 0 = -35 \\
        \det A_x &= (-4)(-2-32)(-1)^{4} + (-7)(-16--24)(-1)^{5} + 0 = 136 + 56 = 192 \\
        \det A_y &= (-4)(-16-1)(-1)^{4} + (-7)(-8-8)(-1)^{5} = 68 - 112 = -44 \\
        \det A_z &= (1)(32--2)(-1)^2 + (2)(-24--16)(-1)^3 + (1)(3--32)(-1)^4 = 34 + 16 + 35 = 85\\
        x &= \frac{192}{-35} ~~~~
        y = \frac{-44}{-35} ~~~~
        z = \frac{85}{-35}
    \end{align*}
     
    }
    \newpage
    \item \Que{
        Cramer's Rule is a good example of why it's the journey, not the destination: It's a terrible way to solve linear systems.
        However, it's important because on the way, you find a way of determining when a system of $n$\ equations of $n$\ unknowns 
        does not have a unique solution.  How can you use Cramer's Rule to predict whether a system of $n$\ equations of $n$\ unknowns
        has a unique solution?
    }
    \Ans{
      If $\det A \not = 0$\ where $A$\ is the coefficient matrix, then $A\mid \Vn{c}$\ has a unique solution (where $\Vn{c}$\ are non-homogeneous constants).
    }
    \end{enumerate}
    ~\\
    \item
        Find the eigenvalues and associated eigenvectors for the following matrices.
    \begin{enumerate}[label=(\alph*)]  
      \item \Que{
      $\begin{Mat}{2} 4 & -15 \\ 2 & -7 \end{Mat}$
      }
      \Ans{
         \begin{align*}
         \det(A-I\lambda)&= \begin{vmatrix} 4-\lambda & -15 \\ 2 & -7-\lambda \end{vmatrix} 
                          = (4-\lambda)(-7-\lambda)-(-15)(2)
                          = \lambda^2 +3\lambda+2 
                          = (\lambda+2)(\lambda+1) 
                          = 0
         \shortintertext{Thus $\lambda_1=-1$\ and $\lambda_2=-2$.}
         \shortintertext{Find eigenvectors using row reduction:}
         (A-\lambda_1 I)\Vn{v}_1 &= \begin{Mat}{2} 5 & -15 \\ 2 & -6 \end{Mat} \to \begin{Mat}{2} 5 & -15 \\ 0 & 0 \end{Mat}
                                \to \begin{Mat}{2} 1 & -3 \\ 0 & 0 \end{Mat} \to
         \Vn{v}_1 = \Ve{3,1} \\
         (A-\lambda_2 I)\Vn{v}_2 &= \begin{Mat}{2} 6 & -15 \\ 2 & -5 \end{Mat} \to \begin{Mat}{2} 6 & -15 \\ 0 & 0 \end{Mat}\to
         \Vn{v}_2 = \Ve{15,6}
         \shortintertext{Thus $\Vn{v}_1=\Ve{3,1}$\ and $\Vn{v}_2=\Ve{15,6}$.}
         \shortintertext{verify:}
         A\Vn{v}_1 &= \begin{Mat}{2} 4 & -15 \\ 2 & -7 \end{Mat} \begin{Mat}{1} 3 \\ 1 \end{Mat} 
                    = \begin{Mat}{1} 12-15 \\ 6 -7 \end{Mat} = \begin{Mat}{1} -3 \\ -1 \end{Mat} 
                    = (-1)\begin{Mat}{1} 3 \\ 1 \end{Mat} = \lambda_1\Vn{v}_1 \\
         A\Vn{v}_2 &= \begin{Mat}{2} 4 & -15 \\ 2 & -7 \end{Mat} \begin{Mat}{1} 15 \\ 6 \end{Mat}
                    = \begin{Mat}{1} 60 - 90 \\ 30 - 42 \end{Mat} = \begin{Mat}{1} -30 \\ -12 \end{Mat} 
                    = (-2)\begin{Mat}{1} 15 \\ 6 \end{Mat} = \lambda_2\Vn{v}_2
         \end{align*}
      }      
      
      \item \Que {
      $\begin{Mat}{2} 18 & -20 \\ 15 & -17 \end{Mat}$
      }
      \Ans{
      \begin{align*}
          \det(A-I\lambda)&= \begin{vmatrix} 18-\lambda & -20 \\ 15 & -17-\lambda \end{vmatrix}
                           = (18-\lambda)(-17-\lambda) - (-20)(15)
                           = \lambda^2 -\lambda - 6
                           = (\lambda-3)(\lambda+2)
                           = 0 
                \shortintertext{Thus $\lambda_1=-2$\ and $\lambda_2=3$.}
                \shortintertext{Find eigenvectors using row reduction:}
                (A-\lambda_1 I)\Vn{v}_1 &= \begin{Mat}{2} 20 & -20 \\ 15 & -15 \end{Mat} \to
                                           \begin{Mat}{2} 1 & -1 \\  0 &   0 \end{Mat} \to
                               \Vn{v}_1  = \Ve{1,1} \\
                (A-\lambda_2 I)\Vn{v}_2 &= \begin{Mat}{2} 15 & -20 \\ 15 & -20 \end{Mat} \to
                                           \begin{Mat}{2}  3 & -4 \\ 0 & 0 \end{Mat}\to
                               \Vn{v}_2  = \Ve{4,3}
                \shortintertext{Thus $\Vn{v}_1 = \Ve{1,1}$\ and $\Vn{v}_2=\Ve{4,3}$.}
                \shortintertext{Verify:}
                A\Vn{v}_1 &= \begin{Mat}{2} 18 & -20 \\ 15 & -17 \end{Mat}\begin{Mat}{1} 1 \\ 1 \end{Mat}
                           = \begin{Mat}{1} 18 - 20 \\ 15 - 17 \end{Mat} = \begin{Mat}{1} -2 \\ -2 \end{Mat}
                           = (-2)\begin{Mat}{1} 1 \\ 1 \end{Mat} = \lambda_1\Vn{v}_1 \\
                A\Vn{v}_2 &= \begin{Mat}{2} 18 & -20 \\ 15 & -17 \end{Mat}\begin{Mat}{1} 4 \\ 3 \end{Mat}
                           = \begin{Mat}{1}  72 - 60 \\ 60 - 51 \end{Mat} = \begin{Mat}{1} 12 \\ 9 \end{Mat}
                           = (3)\begin{Mat}{1} 4 \\ 3 \end{Mat} = \lambda_2\Vn{v}_2              
      \end{align*}
      }
      \item \Que {
      $\begin{Mat}{3} -2 & 5 & 4 \\ 0 & -1 & 0 \\ -2 & 7 & 4 \end{Mat}$
      }
      \Ans{
      \begin{align*}
        \det(A-\lambda I) &= (-1-\lambda)\begin{vmatrix} -2-\lambda & 4 \\ -2 & 4-\lambda \end{vmatrix}(-1)^4
                          = (-1-\lambda)\left[(-2-\lambda)(4-\lambda)-(4)(-2)\right] \\
                         &= (-1-\lambda)(\lambda^2-2\lambda) = -\lambda^2+2\lambda-\lambda^3+2\lambda^2 
                          = -\lambda(\lambda^2-\lambda-2)\\
                         &= -\lambda(\lambda+1)(\lambda-2)=0
               \shortintertext{Thus $\lambda_1 = -1$, $\lambda_2=0$, and $\lambda_3=2$.}
               \shortintertext{Find eigenvectors using row reduction:}
         (A-\lambda_1 I)\Vn{v}_1 &= \begin{Mat}{3} -1 & 5 & 4 \\ 0 & 0 & 0 \\ -2 & 7 & 5 \end{Mat}
                                \to \begin{Mat}{3} 1 & -5 & -4 \\ 0 & -3 & -3 \\ 0 & 0 & 0 \end{Mat}
                                \to \begin{Mat}{3} 3 & -15 & -12 \\ 0 & 15 & 15 \\ 0 & 0 & 0 \end{Mat}
                                \to \begin{Mat}{3} 1 & 0 & 1 \\ 0 & 1 & 1 \\ 0 & 0 & 0 \end{Mat}\\
                        \Vn{v}_1 &= \Ve{1,1,-1}\\
         (A-\lambda_2 I)\Vn{v}_2 &= \begin{Mat}{3} -2 & 5 &  4 \\ 0 & -1 & 0 \\ -2 & 7 & 4 \end{Mat}
                                \to \begin{Mat}{3} -2 & 5 &  4 \\ 0 & -1 & 0 \\  0 & 2 & 0 \end{Mat}
                                \to \begin{Mat}{3}  1 & 0 & -2 \\ 0 &  1 & 0 \\  0 & 0 & 0 \end{Mat} \\
                        \Vn{v}_2 &= \Ve{2, 0, 1}\\
         (A-\lambda_3 I)\Vn{v}_3 &= \begin{Mat}{3} -4 & 5 &  4 \\ 0 & -3 & 0 \\ -2 & 7 & 2 \end{Mat}
                                \to \begin{Mat}{3} -4 & 5 &  4 \\ 0 & -3 & 0 \\  0 & 9 & 0 \end{Mat}
                                \to \begin{Mat}{3}  1 & 0 & -1 \\ 0 &  1 & 0 \\  0 & 0 & 0 \end{Mat} \\
                        \Vn{v}_3 &= \Ve{1,0,1}
         \shortintertext{Verify:}
                       A\Vn{v}_1 &= \begin{Mat}{3} -2 & 5 & 4 \\ 0 & -1 & 0 \\ -2 & 7 & 4 \end{Mat}
                                    \begin{Mat}{1} 1 \\ 1 \\ -1 \end{Mat}
                                  = \begin{Mat}{1} -1 \\ -1 \\ 1 \end{Mat} = (-1)\begin{Mat}{1} 1 \\ 1 \\ -1 \end{Mat}=\lambda_1\Vn{v}_1 \\
                       A\Vn{v}_2 &= \begin{Mat}{3} -2 & 5 & 4 \\ 0 & -1 & 0 \\ -2 & 7 & 4 \end{Mat}
                                    \begin{Mat}{1} 2 \\ 0 \\ 1 \end{Mat}
                                  = \begin{Mat}{1} 0 \\ 0 \\ 0 \end{Mat} = (0)\begin{Mat}{1} 2 \\ 0 \\ 1 \end{Mat} = \lambda_2\Vn{v}_2 \\
                       A\Vn{v}_3 &= \begin{Mat}{3} -2 & 5 & 4 \\ 0 & -1 & 0 \\ -2 & 7 & 4 \end{Mat}
                                    \begin{Mat}{1} 1 \\ 0 \\ 1 \end{Mat}
                                  = \begin{Mat}{1} 2 \\ 0 \\ 2 \end{Mat} = (2)\begin{Mat}{1} 1 \\ 0 \\ 1 \end{Mat} = \lambda_3\Vn{v}_3
      \end{align*}      
      }
      
    \end{enumerate}
    %~\\
    \newpage
    \item
        Let $x_n$\ and $y_n$\ be the number of immature and mature pairs of rabbits at the end of month $n$.  
        If the rabbits breed according to the model of Leonardo of Pisa, we can find $x_{n+1}$\ and $y_{n+1}$\ via
        \begin{align*}
            \begin{Mat}{2} 0 & 1 \\ 1 & 1 \end{Mat} \begin{Mat}{1} x_n \\ y_n \end{Mat} &= \begin{Mat}{1} x_{n+1} \\ y_{n+1} \end{Mat} 
        \end{align*}
    \begin{enumerate}[label=(\alph*)]
    \item \Que{
        Find the eigenvalues and corresponding eigenvectors for the transition matrix.
    }
    \Ans{
        \begin{align*}
          \det(A-\lambda I) &= \begin{vmatrix} 0-\lambda & 1 \\ 1 & 1-\lambda \end{vmatrix} 
                             = (-\lambda)(1-\lambda)-1 
                             = \lambda^2-\lambda-1 
                             = 0
          \shortintertext{Use quadratic equation to find roots:}
          \lambda &= \frac{-b\pm\sqrt{b^2-4ac}}{2a} = \frac{-(-1)\pm\sqrt{(-1)^2-(4)(1)(-1)}}{2(1)} = \frac{1\pm\sqrt{5}}{2} \\
          \lambda_1 &= \frac{1-\sqrt{5}}{2}= 1-\phi ~~~~~~ \lambda_2 = \frac{1+\sqrt{5}}{2} = \phi
          \shortintertext{Find eigenvectors using row reduction:}
          (A-\lambda_1 I)\Vn{v}_1 &= \begin{Mat}{2} -\frac{1-\sqrt{5}}{2} & 1 \\ 1 & 1-\frac{1-\sqrt{5}}{2} \end{Mat} \to
                                     \begin{Mat}{2} 0 & 1+\left(1-\frac{1-\sqrt{5}}{2}\right)\left(\frac{1-\sqrt{5}}{2}\right) \\
                                                    1 & 1-\frac{1-\sqrt{5}}{2} \end{Mat}\\
                                &\to \begin{Mat}{2} 1 & 1-\frac{1-\sqrt{5}}{2} \\
                                                    0 & 1+\frac{1-\sqrt{5}}{2}-\left(\frac{1}{2}-\frac{\sqrt{5}}{2}\right)^2 \end{Mat}
                                 \to \begin{Mat}{2} 1 & 1-\frac{1-\sqrt{5}}{2} \\
                                                    0 & 1 + \frac{1}{2}-\frac{\sqrt{5}}{2} -
                                                    \left(\frac{1}{4} -\frac{\sqrt{5}}{2} + \frac{5}{4}\right) \end{Mat}\\
                                 &\to \begin{Mat}{2} 1 & 1-\frac{1-\sqrt{5}}{2} \\ 0 & 0 \end{Mat} \to 
                          \Vn{v}_1 = \Ve{\frac{1-\sqrt{5}}{2}-1,1} = \Ve{-\frac{1+\sqrt{5}}{2}, 1}\\
          (A-\lambda_2 I)\Vn{v}_2 &= \begin{Mat}{2} -\frac{1+\sqrt{5}}{2} & 1 \\ 1 & 1-\frac{1+\sqrt{5}}{2} \end{Mat} \to
                                     \begin{Mat}{2} 0 & 1+\left(1-\frac{1+\sqrt{5}}{2}\right)\left(\frac{1+\sqrt{5}}{2}\right) \\
                                                    1 & 1-\frac{1+\sqrt{5}}{2} \end{Mat} \\
                                &\to \begin{Mat}{2} 1 & 1-\frac{1+\sqrt{5}}{2} \\
                                                    0 & 1+\frac{1+\sqrt{5}}{2} - \left(\frac{1}{2}+\frac{\sqrt{5}}{2}\right)^2 \end{Mat}
                                 \to \begin{Mat}{2} 1 & 1-\frac{1+\sqrt{5}}{2} \\
                                                    0 & \frac{3+\sqrt{5}}{2} - 
                                                    \left(\frac{1}{4} + \frac{\sqrt{5}}{2} + \frac{5}{4}\right)\end{Mat} \\
                                &\to \begin{Mat}{2} 1 & 1-\frac{1+\sqrt{5}}{2} \\ 0 & 0 \end{Mat} \to
                          \Vn{v}_2 = \Ve{\frac{1+\sqrt{5}}{2}-1,1} = \Ve{-\frac{1-\sqrt{5}}{2},1}
          \shortintertext{Thus $\Vn{v}_1=\Ve{-\phi,1}$\ and $\Vn{v}_2=\Ve{\phi-1,1}$.}
        \end{align*}
    }
    \item \Que{
        Suppose $x_0=1$\ and $y_0=0$.  Express $\begin{Mat}{1} 1 \\ 0 \end{Mat}$\ as a linear combination of the eigenvectors you found.
    }
    \Ans{ 
      \begin{align*}
      \shortintertext{Row reduce:}
          &\begin{Amat}{2} -\frac{1+\sqrt{5}}{2} & -\frac{1-\sqrt{5}}{2} & 1 \\ 1 & 1 & 0 \end{Amat} \to
           \begin{Amat}{2} 1 & 1 & 0 \\ 0 & \frac{1+\sqrt{5}}{2}-\frac{1-\sqrt{5}}{2} & 1 \end{Amat} \to
          \begin{Amat}{2} 1 & 1 & 0 \\ 0 & \sqrt{5} & 1 \end{Amat} \to
           \begin{Amat}{2} 1 & 0 & -\frac{1}{\sqrt{5}} \\ 0 & 1 & \frac{1}{\sqrt{5}} \end{Amat}
      \shortintertext{Verify:}
          &-\frac{1}{\sqrt{5}}\begin{Mat}{1} -\frac{1+\sqrt{5}}{2} \\ 1 \end{Mat} +
            \frac{1}{\sqrt{5}}\begin{Mat}{1} -\frac{1-\sqrt{5}}{2} \\ 1 \end{Mat} =
            \begin{Mat}{1} \frac{\sqrt{5}}{10} + \frac{1}{2} -\frac{\sqrt{5}}{10} +\frac{1}{2} \\
                           \frac{\sqrt{5}}{5}-\frac{\sqrt{5}}{5} \end{Mat}
           =\begin{Mat}{1} 1 \\ 0 \end{Mat}
      \end{align*}
    }
    \newpage
    \item \Que{
        Determine the exact number of immature and mature rabbits at the end of the 12th month ($x_{12}, y_{12}$).
    }
    \Ans{
       \begin{align*}
           \Vn{x}_{12} &= F^{12}\Vn{x}_0 \\
           F^{12}    &= (F^4)(F^4)(F^4) \\
           F^2 &= \begin{Mat}{2} 0 & 1 \\ 1 & 1 \end{Mat}\begin{Mat}{2} 0 & 1 \\ 1 & 1 \end{Mat}
                = \begin{Mat}{2} 1 & 1 \\ 1 & 2 \end{Mat} ~~ 
           F^4  = \begin{Mat}{2} 1 & 1 \\ 1 & 2 \end{Mat}\begin{Mat}{2} 1 & 1 \\ 1 & 2 \end{Mat}
                = \begin{Mat}{2} 2 & 3 \\ 3 & 5 \end{Mat} \\
           F^8 &= \begin{Mat}{2} 2 & 3 \\ 3 & 5 \end{Mat}\begin{Mat}{2} 2 & 3 \\ 3 & 5 \end{Mat}
                = \begin{Mat}{2} 13 & 21 \\ 21 & 34 \end{Mat} ~~
           F^{12}=\begin{Mat}{2} 2 & 3 \\ 3 & 5 \end{Mat}\begin{Mat}{2} 13 & 21 \\ 21 & 34 \end{Mat}
                 =\begin{Mat}{2} 89 & 144 \\ 144 & 233 \end{Mat}\\
           \Vn{x}_{12} &= \begin{Mat}{2} 89 & 144 \\ 144 & 233 \end{Mat}\begin{Mat}{1} 1 \\ 0 \end{Mat} = \begin{Mat}{1} 89 \\ 144 \end{Mat}
       \end{align*}
       Thus at the end of 12 months there are 89 pairs of immature and 144 pairs of mature rabbits.
    }
    \item \Que{
        Use the eigenvalues and eigenvectors to approximate the number of immature and mature at the end of the 12th month.
    }
    \Ans{
       Let $F=\begin{Mat}{2} 0 & 1 \\ 1 & 1 \end{Mat}, \Vn{x}=\begin{Mat}{1} 1 \\ 0 \end{Mat}$, 
       let $\lambda_1,\lambda_2$\ be the eigenvalues and $\Vn{v}_1, \Vn{v}_2$\ be the associated eigenvectors. 
       Expressing $\Vn{x}=\Ve{1,0}$\ in terms of the eigenbasis we get $\Vn{x}^\prime = \Ve{-\frac{1}{\sqrt{5}},\frac{1}{\sqrt{5}}}$.
       \begin{align*}
           \Vn{x}_{12}=F^{12}\Vn{x} &\approx x_1^\prime\lambda_1^{12}\Vn{v}_1 + x_2^\prime\lambda_2^{12}\Vn{v}_2 
                      = -\frac{1}{\sqrt{5}}(1-\phi)^{12}\begin{Mat}{1}-\phi \\ 1 \end{Mat} 
                      +  \frac{1}{\sqrt{5}}\phi^{12}\begin{Mat}{1} \phi-1 \\ 1 \end{Mat} \\
                      &\approx (-0.447)(0.00311)\begin{Mat}{1} -1.618 \\ 1 \end{Mat}
                             + (0.447)(321.997)\begin{Mat}{1} 0.618 \\ 1 \end{Mat}
                      \approx \begin{Mat}{1} 89 \\ 144 \end{Mat}
       \end{align*}
       Thus at the end of 12 months there are approximately $89$\ pairs of immature and $144$\ pairs of mature rabbits.
    }
    
    \end{enumerate}
    ~\\
    
    \item Answer the following questions.
    \begin{enumerate}[label=(\alph*)]
    \item \Que{
        Let $A=\begin{Mat}{4} 2 & 1 & 5 & 1 \\ 0 & 2 & 3 & -1 \\ 1 & -1 & 2 & 1 \end{Mat}$.
        Find a basis for Col($A$) and a basis for Null($A$).
    }
    \Ans{
      \begin{align*}
          \shortintertext{Row reduce:}
          &\begin{Mat}{4} 2 & 1 & 5 & 1 \\ 0 & 2 &  3 & -1 \\ 1 & -1 &  2 & 1 \end{Mat} \to
           \begin{Mat}{4} 1 & 2 & 3 & 0 \\ 0 & 2 &  3 & -1 \\ 0 & -3 & -1 & 1 \end{Mat} \to
           \begin{Mat}{4} 1 & 0 & 0 & 1 \\ 0 & 1 & -2 &  0 \\ 0 &  0 & 7 & -1 \end{Mat} \to
           \begin{Mat}{4} 1 & 0 & 0 & 1 \\ 0 & 7 &  0 & -2 \\ 0 &  0 & 7 & -1 \end{Mat}
           \shortintertext{Col($A$) = Span($\{\Ve{2,0,1},\Ve{1,2,-1},\Ve{5,3,2}\}$)}
           \shortintertext{Parameterize:  Let $7s=w$.}
           x &= -7s\\
           y &= 2s \\
           z &= -s
           \shortintertext{Null($A$) = Span($\Ve{-7,2,-1,7}$)}
      \end{align*}
    }
    \newpage
    \item \Que{
        Mathematicians like to recycle concepts, so the same basic idea will occur in many different places.  
        Remember the range of a function is the set of all possible outputs; 
        Thus we define the range of a linear transformation $T$\ to be the set of all vectors $\Vn{y}$\ for which
        $T\Vn{x}=\Vn{y}$\ for some $\Vn{x}$.  
        Let $T=\begin{Mat}{3} 3 & 1 & 1 \\ -1 & 2 & 1 \\ 1 & 5 & 3 \end{Mat}$.  Find the range of $T$.
    }
    \Ans{
        Row reduce:
        \begin{align*}
            &\begin{Amat}{3} 3 & 1 & 1 & y_1 \\ 
                            -1 & 2 & 1 & y_2 \\ 
                             1 & 5 & 3 & y_3 \end{Amat} \to
             \begin{Amat}{3} 1 & 5 & 3 & y_1+2y_2 \\
                             0 & 7 & 4 & y_2+y_3  \\
                             0 & 0 & 0 & y_3-y_1-2y_2 \end{Amat} \to
             \begin{Amat}{3} 7 & 0 & 1 & 7y_1+9y_2 - 5y_3 \\
                             0 & 7 & 4 & y_2+y_3 \\ 
                             0 & 0 & 0 & y_3-y_1-2y_2 \end{Amat} \to    
        \end{align*}
        Parameterize: Let $x_3=7s$.  Then $x_2=-4s$, and $x_1=-s$.  Thus $\Vn{x}=s\Ve{-1,-4,7}$\ for $s\in\mathbb{R}$.\\
        And thus we see the range of the transformation $T$\ satisfies the equation $-y_1-2y_2-y_3=0$.\\
        Let $y_1=0$, then $y_3=-2y_2$, and $\Vn{y}_1=\Ve{0,1,-2}$.\\
        Let $y_2=0$, then $y_1=-y_3$, and $\Vn{y}_2 =\Ve{-1,0,1}$.\\
        Let $y_3=0$, then $y_1=-2y_2$, and $\Vn{y}_3=\Ve{-2,1,0}$. \\
        Since $y_3 = y_1+2y_2$\ we can see that $\{y_1,y_2\}$\ is independent but $\{y_1,y_2,y_3\}$ is not.\\ 
        Thus the range of $T$ = Span$\{\Ve{0,1,-2},\Ve{-1,0,1}\}$\ which is a plane in $\mathbb{R}^3$.
    }
    
    \end{enumerate}
\end{enumerate} 
\end{document}  